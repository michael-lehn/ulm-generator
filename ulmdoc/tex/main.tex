%------------------------------------------------------------------------------
% Report of the ULM (Ulm Lecture Machine)
% (C) Copyright 2020 by Michael C. Lehn, Andreas F. Borchert
%------------------------------------------------------------------------------

\documentclass[
   a4paper,
   twoside,
   bringhurst,
   palatino,
   english,
   titlepage,
   fleqn
]{refman}

\usepackage{xfp}
\usepackage{tikz,multido,pgfplots}
\usetikzlibrary{arrows,backgrounds,math}
\usetikzlibrary{matrix,shapes,snakes,trees,calc}
\usetikzlibrary{shapes.geometric,shapes.multipart}
\usetikzlibrary{arrows.meta,chains,positioning,quotes}
\usepackage{titlesec}
\usepackage{tikz-qtree}

\usepackage{minitoc}
\usepackage{amsmath}
\setcounter{tocdepth}{1}
\setcounter{minitocdepth}{1}

\pdfextension info {
   /Title   (ULM Ulm Lecture Machine)
}


\makeindex

\titlespacing*{\subsection}{0pt}{9.5ex}{4.3ex}

\begin{document}
   \dominitoc
   \dominilof
   \dominilot

   \begin{titlepage}
   \begin{flushright}
      {\Huge \bfseries
	 Instruction set of the ULM\\
	 (Ulm Lecture Machine) \\
      }

      \vspace*{\fill}

      \href{https://creativecommons.org/licenses/by-sa/4.0/}
      {\includegraphics[width=3cm]{by-sa.pdf}}
      \hspace*{\fill}
      {\huge
	 %Michael C. Lehn \\
	 \today \\
      }
   \end{flushright}
\end{titlepage}

   \cleardoublepage
   \thispagestyle{empty}
   \tableofcontents

    \chapter{Description of the ULM}

    \section{Data Types}

    Binary digits are called \textit{bits} and have the value $0$ or $1$. A
    \textit{bit pattern} is a sequence of bits. For example
    \[
	X := x_{n-1}\dots x_0\;\text{with}\;x_k \in \{0, 1\}\;\text{for}\;0
	\leq k < n
    \]
    denotes a bit pattern $X$ with $n$ bits. The number of bits in bit pattern
    is also called its size or width. The ULM architecture defines a
    \textit{byte} as a bit pattern with 8 bits. Table~\ref{tab.data} lists
    ULM's definitions for \textit{word}, \textit{long word}, \textit{quad word}
    that refer to specific sizes of bit patterns.

    


    \begin{table}[t]
    \begin{center}
    \begin{tabular}{lll}
    \hline
    Data Size & Size in Bytes & Size in Number of Bits \\
    \hline
    Bytes     & -   & 8 \\
    Word      & 2   & 16 \\
    Long Word & 4   & 32 \\
    Quad Word & 8   & 64 \\
    \hline
    \end{tabular}
    \end{center}
    \label{tab.data}
    \caption{Names for specific sizes of bit patterns.}
    \end{table}


    \section{Expressing the Interpretation of a Bit Pattern}

    For a bit pattern $X =  x_{n-1}\dots x_0$ its \textit{unsigned integer}
    value is expressed and defined through
    \[
	u(X) = u(x_{n-1}\dots x_0) := \sum\limits_{k=0}^{n-1} x_k \cdot 2^k
    \]
    \textit{Signed integer} values are represented using the \textit{two's
    complement} and in this respect the notation
    \[
	s(X) = s(x_{n-1}x_{n-2}\dots x_0) :=
	\begin{cases}
	u(x_{n-2}\dots x_0),		& \text{if}\; x_{n-1} = 0, \\
	u(x_{n-2}\dots x_0) - 2^{n-1},	& \text{else} \\
	\end{cases}
    \]
    is used.

    \section{Registers and Virtual Memory}

    The ULM has 256 registers denoted as \%0x00, \dots, \%0xFF. Each of these
    registers has a width of 64 bits. The \%0x00 is a special purpose register
    and also denoted as \textit{zero register}. Reading form the zero register
    always gives a bit pattern where all bits have value 0 (zero bit pattern).
    Writing to the zero register has no effect.


    The (virtual) memory of the ULM is an array of $2^{64}$ memory cells. Each
    memory cell can store exactly one byte. Each memory cell has an index which
    is called its \textit{address}. The address is in the range from $0$ to
    $2^{64-1}$ and the first memory cell of the array has address $0$. In
    notations $M_1(a)$ denotes the memory cell with address $a$.


    \subsection{Endianness}

    For referring to data in memory in quantities of words, long words and quad
    words the definitions
    \[
	\begin{array}{lcl}
	M_2(a) & := & M_1(a) M_1(a+1) \\
	M_4(a) & := & M_2(a) M_2(a+2) \\
	M_8(a) & := & M_4(a) M_4(a+4) \\
	\end{array}
    \]
    are used. The ULM architecture is a \textit{big endian} machine. Therefore
    we have the equalities
    \[
	\begin{array}{lcl}
	u(M_2(a)) & = & u(M_1(a) M_1(a+1)) \\
	u(M_4(a)) & = & u(M_2(a) M_2(a+2)) \\
	u(M_8(a)) & = & u(M_4(a) M_4(a+4)) \\
	\end{array}
    \]


    \subsection{Alignment of Data}

    A quantity of $k$ bytes are aligned in memory if they are stored at an
    address which is a multiple of $k$, i.\,e.
    \[
	M_k(a)\;\text{is aligned}\;\Leftrightarrow\; a \bmod k = 0
    \]


    \chapter{Directives}

    \section{.align <expr>}

    Pad the location counter (in the current segment) to a multiple of
    <expr>.

    \section{.bss}

    Set current segment to the BSS segment.

    \section{.byte <expr>}

    Expression is assembled into next byte.

    \section{.data}

    Set current segment to the data segment.

    \section{.equ <ident>, <expr>}

    Updates the symbol table. Sets the value of <ident> to <expr>.

    \section{.global <ident>}

    Updates the symbol table. Makes the symbol <ident> visible to the linker.


    \section{.globl <ident>}

    Equivalent to \textit{.globl <ident>}:

    Updates the symbol table. Makes the symbol <ident> visible to the linker.

    \section{.long <expr>}

    Expression <expr> is assembled into next long word (4 bytes).

    \section{.space <expr>}

    Emits <expr> bytes. Each byte with value 0x00.

    \section{.string <string-literal>}

    Emits bytes for the zero-terminated <string-literal>.

    \section{.text}

    Set current segment to the text segment.

    \section{.word <expr>}

    Expression <expr> is assembled into next word (2 bytes).

    \section{.quad <expr> }

    Expression <expr> is assembled into next quad word (8 bytes).


    \chapter{Instructions}
%ISA

   \printindex
   % \addcontentsline{toc}{chapter}{Index}
   \nocite{*}
   \bibliography{main}
\end{document}
